\documentclass[12pt]{article}

\usepackage{amsmath}
\usepackage{graphicx}
\usepackage{color}
\usepackage{bbm}
\usepackage{epsfig}
\usepackage{latexsym}
\usepackage{color}


% \RR is supposed to represent the reals.
\newcommand{\RR}{\mathbbm{R}}
% \FF is supposed to represent a field.
\newcommand{\FF}{\mathbbm{F}}
% \CC is supposed to represent the complex plane.
\newcommand{\CC}{\mathbbm{C}}
% \ZZ is supposed to represent the integers
\newcommand{\ZZ}{\mathbbm{Z}}

\newtheorem{theorem}{Theorem}
\newtheorem{lemma}{Lemma}
\newtheorem{corollary}{Corollary}
\newtheorem{conjecture}{Conjecture}

\newcommand{\proof}{\textsc{Proof: }}
% ``eop'' is ``End of Proof''.
\newcommand{\eop}{ $\Box$}

\begin{document}
\title{Tale 2 Cities}
\author{Dickens}
\maketitle
\bibliographystyle{plain}

\begin{abstract}
Best of times, worst of times.
\end{abstract}

% How to add a table of contents:
%\tableofcontents

% How to include an eps file (called ``cell.eps'' in this case).
%\begin{center}
%\epsfig{file=cell.eps,height=3.0in}
%\end{center}

% How to make a figure (with caption, number, etc)
%\begin{figure}[htbp] %  figure placement: here, top, bottom, or page
%   \centering
%   \includegraphics[width=0.9\linewidth]{filename.eps}
%   \caption{ Trenchant description goes here.}
%  \label{fig:my_great_figure}
%\end{figure}

% How to add a bibliography:  Use ``\cite{reference_name}'' and the following
%\bibliography{../../bibliography/master/master}

% If you want the References (i.e. bibliography) to show up as its own
% section in the table of contents, include the following at the top
% of the document, with all the other "usepackage" commands:
%\usepackage[nottoc,numbib]{tocbibind}


\section{Introduction}

Observed: 
\begin{itemize}
\item $N$, the number of subjects in the population
\item Reduction ratio $R_i$ for each subjects (where $R=0.25$ means a 75\% reduction)
\end{itemize}
Latent:
\begin{itemize}
\item $Q_{25} = \Pr[R_{subject}\leq 0.25]$.
\end{itemize}
Then probability of success is:
$$\sum_{k=0}^N \binom{N}{k} Q_{25}^k (1-Q_{25})^{N-k} \mathbbm{1}_{<0.05}(I_{x=0.5}(k, N-k+1))$$



\end{document}
